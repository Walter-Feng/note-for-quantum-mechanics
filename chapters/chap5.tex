\chapter{高等物理化学——量子化学课程笔记}
\section{Linux 具体操作}
\begin{enumerate}
	\item 通过SSH连接所需服务器;
	\item 常用操作、后缀列表:
	\begin{description}
		\item[cd]
			进入输入目录的文件夹;
			`cd amusphere/'
			P.S : `cd ..' 返回上一级文件夹
		\item[ls]
			列出所在文件夹的子文件夹;

		\item[vi]
			打开文件查看内容;

			P.S :输入不存在的文件可直接创立该文件;
		\item[pwd]
			显示所在文件夹;

		\item[mkdir]
			在所在目录下建立文件夹:
			`mkdir levest' 表示建立名字为`levest'的文件夹

		\item[Ins/i]修改文件;`esc'退出修改,':wq!' 保存文件并退出;':q!’不保存退出。
		
		\item[*.log]
			输出文件;

		\item[*.com]
			输入文件/文本文件;

		\item[cp] 复制文件至某个目录
			`cp einfield.com slot' 表示将`einfield.com' 在`slot'文件夹下进行粘贴

		\item[mv] 重命名文件/文件夹(后缀还是要加好);

		\item[rm] remove; `rm file.com' 表示删除该文件;`rm -r rescot' 表示删除该文件夹;

		\item[Shift+G] 移至最后一行;

		\item[gg] 似乎是回到开头;

		\item[tail] 打开文件并至最后一行;

		\item[/`Zero'] 搜索文件含`Zero'的地方;

		\item[formchk] 将chk二进制文件转化;
	\end{description}

\end{enumerate}

\section{调教 Gaussian 步骤}
\begin{enumerate}
	\item 运行GaussView,画出苯甲酸分子;
	\item 点击Calculate $\Rightarrow$ Gaussian Calculation Setup;
	\item 调教各种参数;
	\item 若在windows内部运行,选择后缀为*.gif保存;若在Linux系统,选择后缀为*.com;
	\item 在windows内部运行Gaussian时直接运行;若在Linux系统,
	\begin{itemize}
		\item 修改.com文件中内容:\%chk=benzoicacid.chk
		\item 将脚本文件PBS\_Script 拷贝到输入文件所在文件夹
		\item 输入命令 qsub PBS\_script
		\item 等待。
	\end{itemize}
\end{enumerate}

\section{理论课笔记}
\begin{enumerate}
	\subsection{量子的化学}
	\item 算符对易子:
	\begin{equation}
	[A,B] = AB - BA
	\end{equation}

	\item 算符对易常用公式:
	\begin{align}
	[A,BC]& = [A,B]C - B[A,C]\\
	[AB,C]& = [A,C]B + A[B,C]
	\end{align}

	\item $x$与$p$的对易关系:
	\begin{equation}
	[x,p]\psi=(xp-px)\psi = -i\hbar[x \frac{\partial}{\partial x}\psi-\frac{\partial}{\partial x}(x\psi)]= i\hbar \psi
	\end{equation}

	\item 波函数是量子态在基组中的投影:
	\begin{equation}
	\psi (x) = \bracket{x}{\psi} \quad \psi(p) = \bracket{p}{\psi}
	\end{equation} 

\end{enumerate}

\subsection{量子与经典}

\begin{enumerate}
	\item 一维简谐振子的哈密顿量求解:
	\begin{align*}
	\hat{H}&=\frac{\hat{p}^2}{2m}+\frac{m\omega^2\hat{x}^2}{2}\\
			&=h\omega\frac{m\omega}{2h}(\hat{x}^2+\frac{1}{m^2\omega^2}\hat{p}^2)\\
			&=h\omega\{\sqrt{\frac{m\omega}{2h}}(\hat{x}-\frac{1}{m\omega}\hat{p})\cdot\sqrt{\frac{m\omega}{2h}}(\hat{x}+\frac{1}{m\omega}\hat{p})-\frac{m\omega}{2h}\frac{i}{m\omega}[\hat{x},\hat{p}]\}\\
			&=h\omega(\hat{a}^\dagger \hat{a}+\frac{1}{2})
			\end{align*}

			其中,
			\begin{align}
			&\hat{a}^\dagger=\sqrt{m\omega}{2h}(\hat{x}-\frac{i}{m\omega}\hat{p})\\
			&\hat{a}=\sqrt{m\omega}{2h}(\hat{x}+\frac{i}{m\omega}\hat{p})\\
			&\hat{a}^\dagger\ket{n}=\sqrt{n+1}\ket{n+1}\\
			&\hat{a}\ket{n}=\sqrt{n}\ket{n-1}\\
			&\hat{a}^\dagger\hat{a}\ket{n}=n\ket{n}\\
			\end{align}


	\item 
	\begin{align*}
	\hat{O} & = \sum_i \ket{\phi _i} \bra{\phi_i} \hat{O} \sum_j \ket{\phi _j} \bra{\phi_j} \\
			& = \sum_{ij} \ket{\phi_i} \bracketl{\phi_i}{\hat{O}}{\phi_j}\bra{\phi_j}\\
			& = \sum_{ij} \bracketl{\phi_i}{\hat{O}}{\phi_j} \ket{\phi_i}\bra{\phi_j}\\
			& = \sum_{ij} O_{ij} \ket{\phi_i}\bra{\phi_j}
			\end{align*}

	\item 描述体系状态的本质上是$\ket{\psi}$,波函数只是$\ket{\psi}$在实空间上的投影。

\end{enumerate}

\begin{theorem}
Ehrenfest Theorem:

\begin{align}
\mean{\textbf{p}}= m \frac{d \mean{\textbf{x}}}{dt}\\
\mean{\textbf{F}}= \frac{d\mean{\textbf{p}}}{dt}
\end{align}
\end{theorem}

\begin{proof}
	充分考虑到
	\begin{equation}
	\frac{d\ket{\phi}}{dt}=\frac{1}{i\hbar}[\frac{\textbf{p}^2}{2m}+V(\textbf{x})]\ket{\phi}
	\end{equation}

	\begin{align*}
	\frac{d \mean{\textbf{x}}}{dt}  & = \frac{d \bracketl{\psi}{\textbf{x}}{\psi}}{dt} \\ 
	& = \frac{d\bra{\psi}}{dt} \textbf{x} \ket{\psi} + \bra{\psi} \textbf{x} \frac{d\ket{\psi}}{dt} \\
	&= - \frac{1}{i \hbar}\bra{\psi}[\frac{\textbf{p}^2}{2m}+V(\textbf{x})]\textbf{x}\ket{\psi}+ \bra{\psi}\textbf{x}\frac{1}{i\hbar}[\frac{\textbf{p}^2}{2m}+V(\textbf{x})]\ket{\psi}\\
	&= -\frac{1}{i\hbar}(\frac{\textbf{p}^2}{2m}\textbf{x}-\textbf{x}\frac{\textbf{p}^2}{2m})\ket{\psi}\\
	&= -\frac{1}{2mi\hbar}\bracketl{\psi}{[\textbf{p}^2,\textbf{x}]}{\psi}\\
	&=-\frac{1}{2mi\hbar}\bracketl{\psi}{2\textbf{p}[\textbf{p},\textbf{x}]}{\psi}\\
	&=\frac{1}{m}\bracketl{\phi}{\textbf{p}}{\phi}\\
	&=\mean{\textbf{p}}
	\end{align*}

	\begin{align*}
	\frac{d\mean{\textbf{p}}}{dt} &= \frac{d\bracketl{\psi}{\textbf{p}}{\psi}}{dt}\\
	&=\frac{d\bra{\psi}}{dt} \textbf{p} \ket{\psi} + \bra{\psi} \textbf{p} \frac{d\ket{\psi}}{dt} \\
	&= - \frac{1}{i \hbar}\bra{\psi}[\frac{\textbf{p}^2}{2m}+V(\textbf{x})]\textbf{p}\ket{\psi}+ \bra{\psi}\textbf{p}\frac{1}{i\hbar}[\frac{\textbf{p}^2}{2m}+V(\textbf{x})]\ket{\psi}\\
	&= -\frac{1}{i\hbar}\bracketl{\psi}{[V(\textbf{x}),\textbf{p}]}{\phi} \\
	& = -\frac{1}{i\hbar}\bracketl{\phi}{i\hbar\frac{\partial V(\textbf{x})}{\partial \textbf{x}}}{\phi}\\
	& = \bracketl{\phi}{[-\frac{\partial V(\textbf{x})}{\partial \textbf{x}}]}{\phi}\\
	& = \mean{\textbf{F}}
	\end{align*}

	\begin{align*}
	[V(\textbf{x}),\textbf{p}] \ket{\psi} & = [V(x),-i\hbar \frac{\partial}{\partial \textbf{x}}]\ket{\psi}\\
	& = -i\hbar V(x) \frac{\partial }{\partial \textbf{x}} + i\hbar \frac{\partial}{\partial \textbf{x}}[V(\textbf{x})\ket{\psi}]\\
	&= -i\hbar V(x) \frac{\partial }{\partial \textbf{x}}+ i\hbar \frac{\partial V(\textbf{x})}{\partial \textbf{x}}\ket{\psi} + i\hbar V(\textbf{x}) \frac{\partial \ket{\psi}}{\partial \textbf{x}}\\
	&= i\hbar \frac{\partial V(\textbf{x})}{\partial \textbf{x}} \ket{\psi}
	\end{align*}
\end{proof}

\subsection{量子精确解}

\begin{enumerate}

	\item 拉普拉斯算符极坐标变换:

	\begin{equation}
	\hat{\grad}^2=\frac{1}{r^2}[\frac{\partial}{\partial r}(r^2\frac{\partial}{\partial r})+\frac{1}{\mathrm{sin} \, \theta}\frac{\partial}{\partial \theta}(\mathrm{sin \, \theta}\frac{\partial}{\partial \theta})\frac{1}{\mathrm{sin}^2\,\theta}\frac{\partial^2}{\partial \phi^2}]
	\end{equation}

	\item 分离变量:
	\begin{equation}
	\begin{cases}
	[-\frac{1}{2r^2}\frac{d}{dr}(r^2\frac{d}{dr})-\frac{Z}{r}-E]\ket{\psi_R}=-\frac{\beta}{2r^2}\ket{\psi_{R}}\\
	[-\frac{1}{\mathrm{sin}\, \theta}\frac{\partial}{\partial\theta}(\mathrm{sin}\,\theta\frac{\partial}{\partial \theta})-\frac{1}{\mathrm{sin}^2\,\theta}\frac{\partial^2}{\partial \phi^2}]\ket{\psi_Y}=\beta\ket{\psi_Y}\\
	\\
	\Rightarrow
	\begin{cases}
	\frac{d^2}{d \phi^2}\ket{\psi_\Phi}=-\alpha\\
	\beta\sin^2{\theta}+\frac{\sin{\theta}}{\ket{\psi_\Theta}}\frac{d}{d\theta}(\sin{\theta}\frac{d\ket{\psi_\Theta}}{d\theta})=\alpha
	\end{cases}
	
	\end{cases}
	\end{equation}

	\item 球谐函数:
	\begin{equation}
	Y_{lm}\sqrt{\frac{2l+1}{4\pi}\frac{(l-|m|)!}{(l+|m|)!}P_l^{|m|}(\cos{\theta})e^{im\phi}}\quad 
	\begin{cases}
	l=0,1,2,...,n\\
	m=0,\pm 1,\pm 2,...,\pm l
	\end{cases}
	\end{equation}

	\item 轨道角动量算符:
	\begin{equation}
	\hat{L}^2=-h^2[\frac{1}{\sin{\theta}} \frac{\partial}{\partial \theta}+\frac{1}{\sin^2{\theta}} \frac{\partial^2}{\partial \phi^2}]
	\end{equation}

	\begin{equation}
	\hat{L}^2Y_{lm}=l(l+1)h^2Y_{lm}
	\end{equation}

	\begin{equation}
	\hat{L}_zY_{lm}=mhY_{lm}
	\end{equation}

	\item 径向波函数:
	\begin{equation}
	R_{nl}(r)=\sqrt{(\frac{2Z}{n})^3\frac{(n-l-1)!}{2n[(n+l)!]}}e^{-Zr/n}(\frac{2Zr}{n})^lL^{2l+1}_{n-l-1}(\frac{2Zr}{n})
	\end{equation}
	\begin{equation}
	L_n^\alpha(x)=x^{-\alpha}\frac{1}{n!}(\frac{d}{dx}-1)^nx^{n+\alpha}
	\end{equation}

	\item 类氢原子轨道为理解其他原子电子结构基础,但并非本征,这是由于未考虑电子之间的相互作用,然而完备基使原子轨道的线性组合能够描述分子轨道。


	\item 动能算符的近似描述:
	\begin{equation}
	\hat{T}\psi(x)=-\frac{h^2}{2m}\frac{\partial^2}{\partial x^2}\psi(x)\approx\frac{h^2}{2m\Delta x^2}[2\psi(x)-\psi(x+\Delta x)-\psi(x-\Delta x)]
	\end{equation}

	\begin{equation}
	T_{ij}=\frac{h^2}{2m\Delta x^2}\times
	\begin{cases}
	2,&\quad i=j\\
	-1,&\quad |i-j|=1\\
	0,&\quad others
	\end{cases}
	\end{equation}


	\item 离散变量表象(DVR),优点:形式简单,图像清晰,方法普适。缺点:基组巨大,不适合大量自由度的计算。
	\begin{equation}
	V_{ij}=V(x_i)\delta_{ij}
	\end{equation}
	\begin{equation}
	T_{ij}=\frac{h^2}{2m\Delta x^2}\times
	\begin{cases}
	\frac{\pi^2}{3}, &i=j\\
	\frac{2(-1)^{i-j}}{(i-j)^2},&i\neq j
	\end{cases}
	\end{equation}
\end{enumerate}

\section{量子近似解}

\begin{description}
	\item[变分法] 在有限空间中寻找最优解;任意波函数对应的能量期望值都不小于最低本征态能量

	\begin{theorem}
	The eigenstate of the system has the minimum energy compared to all other states.
	\end{theorem}

	\begin{proof}
	\begin{align}
	&\begin{cases}
	H\ket{\psi_i} = E_i\ket{\psi_i}\\
	\ket{\psi}=\sum_i c_i \ket{\psi_i}
	\end{cases}\\
	\Rightarrow & \bracketl{\psi}{H}{\psi} = \sum_i c_i^* \bracketl{\psi_i}{H \sum_j c_j}{\psi_j} \\
	=& \sum_{ij}c_i^*c_j E_j \delta_{ij}=\sum_i c_i^* c_i E_i \geq \sum_i c_i^* c_i E_{0}=E_{0}
	\end{align}
	\end{proof}

	\item[微扰法] 尽可能利用简单体系的精确解描述复杂问题
	\begin{equation}
	H\ket{\psi_i}=E_i\ket{\psi_i} \quad H=H^{(0)} +\lambda H^{(1)}
	\end{equation}

	\begin{align*}
	&\begin{cases}
	\ket{\psi_i} = \ket{\psi_i^{(0)}}+\lambda \ket{\psi_i^{(1)}}\\
	E_i=E_i^{(0)} + \lambda E_i^{(1)}
	\end{cases}\\
	\Rightarrow & (H^{(0)}+\lambda H^{(1)})(\ket{\psi^{(0)}_i}+\lambda \ket{\psi_i^{(0)}})=(E_i^{(0)}+\lambda E^{(1)}_i)(\ket{\psi_i^{(0)}}+\lambda \ket{\psi^{(1)}_i}) \\
	\Rightarrow & \begin{cases}
	(H^{(0)}-E_i^{(0)})\ket{\psi_i^{(0)}}=0\\
	(H^{(0)}-E_i^{(0)})\ket{\psi_i^{(1)}}+(H^{(1)}-E_i^{(1)})\ket{\psi_i^{(0)}} =0
	\end{cases}\\
	\Rightarrow & \bracketl{\psi_j^{(0)}}{H^{(0)}-E_i^{(0)}}{\psi_i^{(1)}}=\bracketl{\psi_j^{(0)}}{(E_i^{(1)}-H^{(1)})}{\psi_i^{(0)}}\\
	\Rightarrow & (E_j^{(0)}-E_i^{(0)})\bracket{\psi_j^{(0)}}{\psi_i^{(1)}}=E_i^{(1)}\delta_{ij} -\bracketl{\psi^{(0)}_j}{H^{(1)}}{\psi_i^{(0)}}\\
	\Rightarrow &
	\begin{cases}
	E_i^{(1)}=\bracketl{\psi_i^{(0)}}{H^{(1)}}{\psi_i^{(0)}} &\quad i=j \\
	\ket{\psi_i^{(1)}}=\sum_{j\neq i}\frac{\bracketl{\psi^{(0)}_j}{H^{(1)}}{\psi^{(0)}_i}}{E^{(0)}_i-E^{(0)}_j} &\quad i\neq j
	\end{cases}
	\end{align*}

	二阶微扰先略。

	微扰应用:

	\begin{equation}
	T=\frac{p^2}{2m}-\frac{p^4}{8c^2m^3}+....
	\end{equation}
	\begin{description}
		\item 非谐性:对势能项做微扰
		\begin{equation}
		V=V(0)+\frac{k}{2}x^2+\frac{\alpha}{6}x^3+\frac{\beta}{24}x^4+...
		\end{equation}
		\item Stark效应:对外场做微扰
		\begin{equation}
		H=H_0-\mu\cdot F
		\end{equation}
	\end{description}

		
\end{description}


\begin{description}
	\item[全同性原理] 量子世界的全同粒子不可区分,任何两个粒子交换不影响体系的状态;

	\item[交换算符]
	\begin{equation}
	\textbf{P}_{12}[f(q_1,q_2,....,q_n)]=f(p_2,p_1,....,p_n)
	\end{equation}
	\begin{equation}
	P_{12}[1s(1)\alpha(1)3s(2)\beta(2)]=1s(2)\alpha(2)3s(1)\beta(1)
	\end{equation}

	\item[交换算符的本征值]
	\begin{equation}
	\textbf{P}_{12}[\textbf{P}_{12}[f(q_1,q_2,.....,q_n)]]=f(q_1,q_2,...,q_n)\Rightarrow P_{12}^2=1
	\end{equation}
	Thus
	\begin{equation}
	c_i=\pm 1
	\end{equation}

	交换算符的本征值为实数,只能为$\pm 1$,对应波函数分别称为对称和反对称。不具备对称性的波函数无法用于描述全同粒子。

	\item[玻色子和费米子] 波函数反对称的粒子为费米子,而波函数对称的粒子为玻色子

	\item[反对称粒子的泡利不相容原理]
	\begin{equation}
	f(q_1,q_1,....,q_n)=-f(q_1,q_1,....,q_n) \Rightarrow f(q_1,q_1,....,q_n)=0
	\end{equation}
	\item[自旋统计] Pauli 使用量子场论证明了自旋统计理论,指出费米子(电子和质子等)拥有半整数自旋,而玻色子(光子和声子等)具有整数自旋,电子具有$\frac{1}{2}$的自旋。

	\item[相对论量子化学] 自旋角动量很自然地以内禀方式蕴含在相对论性狄拉克方程中。作为最低阶非相对论近似,薛定谔方程人为地丢弃了自旋这种相对论效应。

	\item[Slater行列式]
	\begin{equation}
	|\Psi(p_1,p_2,...,p_n)|=\frac{1}{\sqrt{N!}}
	\begin{vmatrix}
	\ket{\phi_1(q_1)} & \ket{\phi_2(q_1)} & ... & \ket{\phi_n(q_1)}\\
	\ket{\phi_2(q_2)} & \ket{\phi_2(q_2)} & ... & \ket{\phi_n(q_2)}\\
	... & ...& ... & ...\\
	\ket{\phi_1(q_n)} & \ket{\phi_2(q_n)} & ... & \ket{\phi_n(q_n)}\\
	\end{vmatrix}
	\end{equation}

	\item[双电子自旋]
	\begin{equation}
	\alpha(1)\alpha(2),\beta(1)\beta(2),\alpha(1)\beta(2),\beta(1)\alpha(2)
	\begin{cases}
	\frac{1}{\sqrt{2}}[\alpha(1)\beta(2)+\beta(1)\alpha(2)]\\
	\frac{1}{\sqrt{2}}[\alpha(1)\beta(2)-\beta(1)\alpha(2)]
	\end{cases}
	\end{equation}

\end{description}

\begin{align*}
E&=\frac{\bracketl{\psi}{\hat{H}}{\psi}}{\bracket{\psi}{\psi}}\\
&=\frac{(\sum_ic_i^*\bra{\phi_i})\hat{H}(\sum_jc_j\ket{\phi_j})}{\sum_ic_i^*\bra{\phi_i}(\sum_jc_j\ket{\phi_j})}\\
&=\frac{\sum_{ij}c_i^*c_j\bracketl{\phi_i}{\hat{H}}{\phi_j}}{\sum_{ij}c_i^*c_j\bracket{\phi_i}{\phi_j}}\\
&=\frac{\sum_{ij}c_i^*H_{ij}c_j}{\sum_{ij}c_i^*S_{ij}c_j}\\
&=\frac{C^*HC}{C^*SC}\\
&\Rightarrow EC^*SC=C^*HC\\
&\Rightarrow \partial_{c_k}(EC^*SC)=\partial_{c_k}(C^*HC)\\
&\Rightarrow (\partial_{c_k}E) C^*SC+E(\partial_{c_k}C^*)SC+EC^*S\partial_{c_k}C=(\partial_{c_k}C^*)HC+C^*H\partial_{c_k}C\\
&\Rightarrow \partial_{c_k}E = \frac{\partial_{c_k}(HC-ESC)+(C^*H-EC^*S)\partial_{c_k}C}{C^*SC}
\end{align*}
\begin{align*}
&\frac{\partial E}{\partial c_k}=0\\
&\Rightarrow 
\begin{cases}
HC-ESC=0\\
C^*H-EC^*S=0
\end{cases}\\
&\Rightarrow HC=ESC
\end{align*}
\begin{align}
&\begin{cases}
\ket{\phi_{1s}^{(1)}}=\frac{1}{\sqrt{\pi}}e^{-r_1};\\
\ket{\phi_{1s}^{(1)}}=\frac{1}{\sqrt{\pi}}e^{-r_2};
\end{cases}\\
\Rightarrow& S=\bracket{\phi_{1s}^{1}}{\phi_{1s}^{(2)}}=(1+R+\frac{R^2}{3})e^{-R}
\end{align}

\begin{align*}
\bracketl{\phi_{1s}^{(1)}}{\hat{H}}{\phi_{1s}^{(1)}}&=\bracketl{\phi_{1s}^{(1)}}{(-\frac{\grad_e^2}{2}-\frac{1}{r_1}-\frac{1}{r_2}+\frac{1}{R})}{\phi_{1s}^{(1)}}\\
&=\bracketl{\phi_{1s}^{(1)}}{-\frac{\grad^2_e}{2}-\frac{1}{r_1}}{\phi_{1s}^{(1)}}-\bracketl{\phi_{1s}^{(1)}}{\frac{1}{r_2}}{\phi_{1s}^{(1)}}+\frac{1}{R}\\
E_{1s}-\bracketl{\phi_{1s}^{(1)}}{\frac{1}{r_2}}{\phi_{1s}^{(1)}}
&=E_{1s}+J
\end{align*}


\begin{equation*}
\bracketl{\phi_{1s}^{(2)}}{\hat{H}}{\phi_{1s}^{(2)}}=E_1s+J
\end{equation*}

\begin{description}
	\item[库伦积分法]
	\begin{equation}
	\begin{cases}
	\ket{\phi_{1s}^{(1)}}=\frac{1}{\sqrt{\pi}}e^{-r_1}\\
	\ket{\phi_{1s}^{(2)}}\frac{1}{\sqrt{\pi}}e^{-r_2}
	\end{cases}
	\Rightarrow J=(1+\frac{1}{R})e^{-2R}
	\end{equation}

	\item[交换积分]
	\begin{equation}
	\begin{cases}
	\ket{\phi_{1s}^{(1)}}=\frac{1}{\sqrt{\pi}}e^{-r_1}\\
	\ket{\phi_{1s}^{(2)}}\frac{1}{\sqrt{\pi}}e^{-r_2}
	\end{cases}
	\Rightarrow K=(\frac{1}{R}-\frac{2R}{3})e^{-R}
	\end{equation}
	它在分子成键中起到了非常关键的作用,是量子效应。
\end{description}

\begin{align*}
& \ket{\psi(c_1,c_2)} = c_1\ket{\psi_{1s}^{(1)}} + c_2 \ket{\psi_{1s}^{(2)}}\\
\Rightarrow & E(c_1,c_2)= \frac{\bracketl{\psi_(c_1,c_2)}{\hat{H}}{\psi(c_1,c_2)}}{\bracket{\psi(c_1,c_2)}{\psi(c_1,c_2)}}\\
=&\frac{(c_1^2+c_2^2)(E_{1s}+J)+(c_1^*c_2+c_1c_2^*)(SE_{1s}+K)}{(c_1^2+c_2^2)+(c_1^*c_2+c_1c_2^*)S}\\
=& (E_{1s}+J)+\frac{(c_1^*c_2+c_1c_2^*)(K-SJ)}{(c_1^2+c_2^2)+(c_1^*c_2+c_1c_2^*)S}\\
=&(E_{1s}+J)+\frac{K-SJ}{\frac{c_1^2+c_2^2}{c_1^*c_2+c_1c_2^*}+S}
\end{align*}

\begin{description}
	\item[平均场近似] 忽略电子动态关联,多电子问题转化为单电子问题,多电子波函数简化为单电子波函数的乘积,使用自洽场得到单电子波函数

	\item[Koopmans 定理]假设离子轨道与中性分子轨道相同,可计算分子的电离能喝亲和势,改变电子数会引起单电子轨道重排和弛豫,与电子关联量级相似,但符号相反
\end{description} 
\begin{align*}
&\mean{\hat{H}}_{Hartree} \\
=& \bracketl{\Psi(\{r_i\})}{\hat{H}}{\Psi(\{r_i\})}\\
=&(\prod_i\bra{\phi_i(r_i)})(\sum_k\hat{h}_k+\frac{1}{2}\sum_{k\neq l}\frac{1}{\hat{r}_kl})(\prod_j\ket{\phi_j(r_j)})\\
=&\sum_k\bracketl{\phi_k(r_k)}{\hat{h}_k}{\phi_k(r_k)}(\prod_{i\neq k}\bra{\phi_i(r_i)})(\prod_{j \neq k}\ket{\phi_j(r_j)})\\
&+\frac{1}{2}\sum_{k \neq l}\bracketl{\phi_k(r_k)\phi_l(r_l)}{\frac{1}{\hat{r}_{kl}}\phi_k(r_k)\phi_l(r_l)}(\prod_{i \neq \{k,l\}}\bra{\phi_i(r_i)})(\prod_{i \neq \{k,l\}}\ket{\phi_j(r_j)})\\
=&\sum_k\bracketl{\phi_k(r_k)}{\hat{h}_k}{\phi_k(r_k)}+\frac{1}{2}\sum_{k \neq l}\bracketl{\phi_k(r_k)\phi_l(r_l)}{\frac{1}{\hat{r}_{kl}}}{\phi_k(r_k)\phi_l(r_l)}
\end{align*}

\begin{description}
	\item[变分法与基组] Hartree-Fock 变分法中,波函数的形式原则上是任意的,但是波函数的随意变化数字处理起来非常麻烦,使用基组对波函数进行线性展开,对量子化学中的所有电子结构计算都至关重要。

	\item[原子轨道基组] 完备的基组在量子化学是很难实现的,从计算量的角度需要尽可能地减少基组的数目。氢原子的电子结构可解析求解,为理解更复杂原子和分子的电子结构打开了大门。使用氢原子和类氢离子的原子轨道作为基组,是解决复杂化学问题的通用做法,有着重要的意义。

	\item[原子轨道]
	\begin{equation}
		\psi(r,\theta,\phi)=R_{nl}(r)Y_{lm}(\theta,\phi)
	\end{equation}

	\item[Slater基组和Gaussian基组]
	\begin{equation}
	\chi(r)\sim e^{-\xi r} \qquad g(r)\sim e^{-\alpha r^2}
	\end{equation}

	从基组数目角度看,STO比GTO好;

	从波函数积分计算效率角度看,GTO比STO好。

	\item[高斯函数线性组合]
	\begin{equation}
	e^{-\xi r}=\frac{\xi}{2\sqrt{\pi}}\int_0^\infty \alpha^{-3/2} e^{-\xi^2/4\alpha}e^{-\alpha r^2} \, d\alpha 
	\end{equation}
	\begin{equation}
	\Rightarrow \xi_\mu(r-R_A)\approx \sum_Pc_{p\mu}g_p(\alpha_{p\mu},r-R_p)
	\end{equation}
	\begin{equation}
	\begin{cases}
	g_{1s}(\alpha,r)=(8\alpha^3/\pi^3)^{1/4}e^{-\alpha r^2};\\
	(\alpha,r)=(128\alpha^5/\pi^3)^{1/4}xe^{-\alpha r^2};\\
	g_{3dxy}(\alpha,r)=(2048\alpha^7/\pi^3)^{1/4}xye^{-\alpha^2}
	\end{cases}
	\end{equation}
\end{description}

\begin{description}
	\item[Dunning 原子轨道基组] 如 aug-cc-pVDZ, aug-cc-pVTZ, aug-cc-pVQZ, aug-cc-pV5Z...

	aug:弥散, cc: correlation consistent; p: polarized; V: Valence bonds; DZ: double zeta.

	\item[赝势原子轨道基组] 当原子中的电子很多事,内层电子通常对原子性质的影响很小,但是其对应的全电子基组很复杂,为了减少计算量,其贡献可通过有效势能来描述,而且可以包含相对论效应的修正。

	\item[赝势基组]
	赝势基组包括赝势和基组两部分,内部电子的贡献采用赝势描述,直接放在哈密顿量里,外层价电子采用一般的基组。

	LANL1只考虑加电子,LANL2系列除了加电子外,还考虑次外层电子,因为他们与价层的能查不明显,而且对成键有贡献。

	LanL2DZ:对H-Ne使用D95V全电子基组,对Na-Bi使用赝势基组,也就是LANL有效核势加上DZ基组;LanL2DZ是常用基组,适合过渡金属等中等质量的金属元素。

	\item[总能量] 变分法保证基组越大,HF能量越小,越接近于精确值。

	\item[Hessian矩阵]
	\begin{equation}
	F_{ij}=\frac{1}{\sqrt{m_im_j}}\frac{\partial^2 U}{\partial x_i \partial x_j}
	\end{equation}

	\item[振动频率]
	\begin{equation}
	\sum_{j-1}^N (F_{ij}-\delta_{ij}\lambda_k)l_{jk}\Rightarrow det(F_{ij}-\delta_{ij}\lambda_{k})=0
	\end{equation}
\end{description}

\begin{description}
	\item[电子关联] Hartree-Fock理论中采用平均场处理电子-电子相互作用,缺乏电子关联,直接用多电子基组展开多电子波函数计算量巨大,很难实现。

	\item[如何有效加入电子关联] 以Hartree-Fock得到的单Slater行列式为基础,考虑多电子关联的主要方法有CI(Configurate Interaction)、MCSCF(Multi-Configured Self-Consistent Field)、MPn、CC(Couple Cluster)等。

	HF方法的计算量与电子数的四次方成正比,MP2为五次方,CISD和CCS为六次方,CCSD(T)为七次方,CISDT和CCSDT为八次方,随着电子数增加而迅速增加。

	\item[多电子组态] 在Hartree-Fock分子轨道的基础上,组态相当于多电子组态,用之可展开多电子波函数,对角化多电子哈密顿量,得到基态和激发态性质。

	若有K个分子轨道,每个轨道最多占据两个电子,总电子书数N,则n激发组态数为
	\begin{equation}
	C_N^nC_{2K-N}=\frac{N!}{n!(N-n)!}\frac{(2K-N)!}{n!(2K-N-n)!}
	\end{equation}

	每个组态都可以用Slater行列式表示。


	\item[多电子波函数的组态线性组合] 
	\begin{align*}
	\ket{\Phi} &= c_0\ket{\psi_0} +\sum_{ra}c_a^r\ket{\psi_a^r}+\sum_{a<b,r<s}c_{ab}^{rs}\ket{\psi_{ab}^{rs}}+\sum_{a<b<c,r<s<t}c^{rst}_{abc}\ket{\psi^{rst}_{abc}}+...\\
	&=c_0\ket{\psi_0} + c_S\ket{S}+c_D\ket{D}+c_T\ket{T}+...
	\end{align*}

	Full CI(FCI)包括所有可能的激发,可以给出多电子问题的精确解,

	实际上即使我们只考虑有限个单电子基组,所有可能的N-电子基组数目也非常庞大。

	通常需要对组态进行阶段,只处理有限个N-电子基组。如只考虑单激发,成为CIS;截断到双激发,成为CISD。
	

	\item[组态相互作用关系] 单激发对基态能量没有直接贡献,双激发对基态能量的修正起首要作用,Hartree Fock的本征值问题等价于确保组态与单激发不直接混合,相差两个激发的组态之间没有直接相互作用。

	\item[Brillouin 定理]
	\begin{equation}
	\bracketl{\psi_0}{\hat{H}}{\psi_a^r}=\bracketl{a}{\hat{h}}{r}+\sum_b\bra{ab}\ket{rb}=\bracketl{\phi_a}{\hat{f}}{\phi_r}=0
	\end{equation}

	\begin{align*}
	\bracketl{\psi_0}{\hat{H}}{\psi_a^r}&=(\prod_i\bra{\phi_i})O_1(\prod_j\ket{\phi_j})\\
	&=\sum_k(\prod_i\bra{\psi{\phi_i}})h_k(\prod_l\ket{\phi_j})\\
	&=\sum_k \bracketl{\phi_k}{h_k}{\phi_k}*\prod_{i \neq k} \bra{\phi_i} \prod_{j \neq k} \bra{\phi_j}\\
	&-\sum_k \bracketl{\phi_k}{h_k}{\phi_k}
	\end{align*}
\end{description}

\begin{description}
	\item[Thomas-Fermi 理论]
		是现代密度泛函理论的雏形,能量期望值完全形成电子密度的泛函,使得电子结构处理变得非常简单。

		当时其精度有限,因为动能项通过简单的近似得到,而且没有考虑交换和关联对总能量的贡献。
	\item[Hohenberg-Kohn 定理]
	\begin{itemize}
		\item 所有客观测量的期望值原则上都是基态电子密度的泛函。

		电子密度分布$\rho(r)$决定了总电子数$N$,从而哈密顿量$H=T_e+U_{ee}+V_{ext}$总只有最后一项可能是不确定的。如果$\rho(r)$是两个不同哈密顿量$H$和$H'$的基态电子密度,而两者的基态波函数分别为$\ket{\Psi}$和$\ket{\Psi'}$,基态能量分别为$E_0$和$E'_0$,基态在不兼并的情况下有
		\begin{equation}
		\begin{cases}
		E_0&=\bracketl{\Psi}{H}{\Psi}\\
		&<\bracketl{\Psi'}{H}{\Psi'}\\
		&=\bracketl{\Psi'}{H'}{\Psi'}+\bracketl{\Psi'}{(H-H')}{\Psi'}\\
		&=E_0'+\int \rho(r)(V_{ext}-V'_{ext})\,dr\\
		E_0'&=\bracketl{\Psi'}{H'}{\Psi'}\\&
		<\bracketl{\Psi}{H'}{\Psi}\\
		&=\bracketl{\Psi}{H}{\Psi}+\bracketl{\Psi}{(H'-H)}{\Psi}\\
		&=E_0-\int \rho(r)(V_{ext}-V'_{ext})\,dr\\
		\end{cases}
		\end{equation}
		Then
		\begin{equation}
		E_0+E'_0<E_0+E_0'
		\end{equation}
		矛盾,故电子密度与唯一的哈密顿量对应。反过来,哈密顿量有唯一的基态电子密度。\footnote{在这里,需要指出的是此定理在基态不兼并的情况下成立。}因此,所有可观测量的期望值都是基态电子密度的泛函,特别地,基态总能量是电子密度的泛函:
		\begin{equation}
		E[\rho]=T_e[\rho]+V_{ext}[\rho]+U_{ee}[\rho]
		\end{equation}

		\item 基态电子密度原则上可以通过变分法严格得到。

	\end{itemize}

	\item[Kohn-Sham密度泛函理论]
	哈密顿量中的动能项难以用密度泛函直接表达,而波函数可以更容易地计算动能,为此1965年Kohn和Sham提出了融合波函数和密度的现代DFT理论,使密度泛函理论成为实际可行的理论方法。

	\begin{itemize}
		\item 动能 
		\begin{equation}
		T_0[\rho]=\sum_i\bracketl{\phi_i}{-\frac{1}{2}\grad^2}{\phi_i}
		\end{equation}

		\item 库伦势
		\begin{equation}
		U_{cl}[\rho]=\frac{1}{2}\iint\frac{\rho(r')\rho(r)}{|r'-r|}\,drdr'
		\end{equation}
	\end{itemize}

\end{description}

\begin{align*}
&H\ket{\phi_i}=E_i\ket{\phi_i}\\
&\ket{\phi(t=0)}\sum_ic_i(0)\ket{\phi_i}\\
\Rightarrow & \frac{\partial \ket{\phi(t)}}{\partial t}=\frac{1}{i\hbar}H\ket{\phi(t)}\\
\Rightarrow & c_i\ket{\phi_i}=\frac{1}{i\hbar}\sum_i\bracket{H}{\phi_i}\\
\Rightarrow & \bracketl{\phi_j}{\sum_ic_i}{\phi_i}=\frac{1}{i\hbar}\sum_ic_iE_i\bracket{\phi_j}{\phi_i}\\
\Rightarrow & c_j^*=\frac{1}{i\hbar}c_jE_j\\
\Rightarrow & c_j(t)=c_j(0)e^{\frac{e_j}{i\hbar}t}\\
\Rightarrow & \ket{\phi(t)}=\sum_ic_j\ket{\phi_j}
\end{align*}

\begin{description}
	\item[HK定理] 体系的基态电子密度与外势一一对应;
	\item[RG定理] 但体系的初态确定时,任何时刻电子密度都与外势一一对应;
	\begin{description}
	\item[正定理] 哈密顿量
	\begin{equation}
	H(r,t)=T_{e}(r)+U_{ee}(r)+V_{ext}(r,t)
	\end{equation} 
	其中,动能项 $T_e=-\sum\frac{\grad^2_i}{2}$,电子相互作用项 $U_{ee}=\frac{1}{2}\sum_{i\neq j}\frac{1}{r_i-r_j}$
	当外势项$V_{ext}(r,t)=\sum_iv(r,t)$已知时,含时哈密顿量确定。
	
	\item[外势有唯一的含时电子密度与之对应]
	薛定谔方程 $i\frac{\partial}{\partial t}\ket{\psi(r,t)}=H(r,t)$

	\item[逆定理] 假设

	\item[电流密度] 
	\begin{equation}
	j(r)=\frac{1}{2i}\sum_{i=1}^{N}[\grad_i\delta(r-r_i)+\delta(r-r_i)\grad_i]\quad f(r,t)=\bracketl{\psi(t)}{j(r)}{\psi(t)}\end{equation}


	 \item[海森堡方程]
	 \begin{equation}
	 \frac{\partial}{\partial t}j(r,t)=-i\bracketl{\psi(r,t)}{[j(r)H(r,t)]}{\psi(r,t)}
	 \end{equation}


	\end{description}

	分子存在振动能级,其能级差正好落于红外光波段。外界电磁波对体系的影响可通过哈密顿量的一阶微扰进行处理:
\begin{equation}
\hat{H}=\hat{H}_0 + \hat{W}
\label{pertubation}
\end{equation}
$\ket{\phi_n}$定义为基态哈密顿量,即$\hat{H}_0$的特征向量。由于$\hat{H}_0$是一个可观测量,其特征向量能够形成完备基组,故针对\ref{pertubation}的特征向量$\ket{\psi(t)}$可进行线性展开:
\begin{equation}
\psi(t)=\sum_nc_n(t)\,\mathrm{exp}(-\frac{i}{\hbar}E_nt)\,\ket{\phi_n}
\label{linearexpansion}
\end{equation}
其中$E_n$为$\ket{\phi_n}$对应的特征根。并且拥有初始条件
\begin{equation}
c_n(t=0)=\delta_{n,i}
\end{equation}
其中,$i$表示$t=0$状态下体系所处的状态。当体系经过$t$时间后,体系被观测为处于$f$的状态的概率为
\begin{equation}
w_{i\rightarrow f}(t)=|c_f(t)|^2
\end{equation}
为计算$c_n$对\ref{linearexpansion}式对时间进行求导,并结合\sch's equation,得到
\begin{equation}
i\hbar\sum_n(\frac{dc_n}{dt}-\frac{i}{\hbar}E_nc_n)\,\mathrm{exp}(-\frac{i}{\hbar}E_nt)\ket{\phi_n}=\sum_nc_n\mathrm{exp}(-\frac{i}{\hbar}E_nt)(E_n+\hat{W})\ket{\phi_n}
\end{equation}
化简并左乘$\bra{\phi_m}$,得到
\begin{equation}
i\hbar\frac{dc_m}{dt}=\sum_nW_{mn}c_n\,\mathrm{exp}[\frac{i}{\hbar}(E_m-E_n)t]
\end{equation}
并定义
\begin{equation}
W_{mn}=\bracketl{\phi_m}{\hat{W}}{\phi_n}
\end{equation}
对公式进行积分,可得到
\begin{equation}
c_m(t)=c_m(0)+\frac{1}{i\hbar}\int^t_0dt'\sum_nW_{mn}\,\mathrm{exp}[\frac{i}{\hbar}(E_m-E_n)t']c_n(t')
\end{equation}
针对微扰的一阶项,$c_m(t)$由下式给出:
\begin{equation}
c_m(t)=c_m(0)+\frac{1}{i\hbar}\int^t_0dt'\sum_nW_{mn}\,\mathrm{exp}[\frac{i}{\hbar}(E_m-E_n)t']c_n(0)
\end{equation}
代入状态$i$及$f$的初始条件,得到
\begin{equation}
c_f(t)=\frac{1}{i\hbar}\int^t_0dt'W_{fi}\,\mathrm{exp}[\frac{i}{\hbar}(E_f-E_i)t']
\end{equation}
并假设$W_{fi}$并不随时间发生改变,从而直接积分得到
\begin{equation}
|c_f(t)|^2=w_{i\rightarrow f}(t)=|W_{fi}|^2\frac{\mathrm{sin}^2[(E_f-E_i)t/(2\hbar)]}{[(E_f-E_i)/2]^2}
\end{equation}
若对一个较大的$t$,有
\begin{equation}
w_{i\rightarrow f}(t)\approx |W_{fi}|^2\frac{2\pi}{\hbar}t\delta(E_f-E_i)
\end{equation}
即每一个元时间发生跃迁的概率为
\begin{equation}
P_{i\rightarrow f}=\frac{1}{t}w_{i\rightarrow f}(t) = \frac{2\pi}{\hbar}|W_{fi}|^2\delta(E_f-E_i)
\end{equation}
从而推导得到Fermi's Golden Rule.
\begin{equation}
P_{i\rightarrow f}=\frac{2\pi}{\hbar}|\bracketl{\phi_f}{\hat{W}}{\phi_i}|^2\rho(E_f=E_i)
\end{equation}
其中$\rho(E_f=E_i)$表达了其态密度。

在偶极近似中,与电磁波的作用可表示为
\begin{equation}
\hat{V}(t)=\hat{V}_{el}(t)+\hat{V}_{mag}(t)
\begin{cases}
\hat{V}_{el}(t)=-\overrightarrow{\mu}_{el}\cdot \overrightarrow{E}\\
\hat{V}_{mag}(t)=-\overrightarrow{\mu}_{mag}\cdot \overrightarrow{E}\\
\end{cases}
\end{equation}
在低磁场环境下电场相互作用占主导,从而推导得到吸收峰
\begin{equation}
G_{fi}=\frac{8\pi^3h}{3h^2(4\pi\varepsilon_0)c}|\bracketl{f}{\overrightarrow{\mu}}{i}|^2
\end{equation}
从而可得到$G_{fi}$与偶极矩大小成正相关,若一个键的振动带来大的偶极矩变化,其吸收峰会较大;而若不产生偶极矩,则$|\bracketl{f}{\overrightarrow{\mu}}{i}|$一项为零,故不发生吸收。
\end{description}   




