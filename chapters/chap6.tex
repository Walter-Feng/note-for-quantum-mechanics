\section{组会笔记}
\begin{description}
	\item[离散空间连续化]
		可以理解为给一大堆的$\delta$函数赋予一定的宽度,或者将一堆离散的点进行拟合;

	\item[连续空间离散化]将函数用许多高斯函数进行分离,然后将高斯函数转化为$\delta$函数;

	\item[Charging Patching Method] 假设一些部分体系之间的相互作用是有限的,从而能够将一些大体系划分为几个小体系计算电子密度(个人猜测其内涵)

	\begin{quote}
	It is assumed that the charge density at a given point depends only on the local atomic environment around the point. As a relsult, charging density motifs for all the atoms can be calculated from prototype system, and can then be used to reassemble the charge density of a large system.
	\end{quote}

	\begin{equation}
	m_{l_\alpha} (\textbf{r}-\textbf{R}_\alpha) = \rho_{LDA}(\textbf{r}) \frac{w_\alpha(|\textbf{r}-\textbf{R}_\alpha|)}{\sum_{\textbf{R}_{\alpha'}}w_{\alpha'}(|\textbf{r}-\textbf{R}_{\alpha'}|)}
	\end{equation}

	\footnotesize

	$R_\alpha$: an atomic site of atom typed $\alpha$

	$m_{l_\alpha}(\textbf{r}-\textbf{R}_\alpha)$: charge-density motif belonging to this atomic site;

	$\rho_{LDA}(\textbf{r})$: the self-consistently calculated charge density of a prototype system

	$w_\alpha(\textbf{r})$: exponential decay function

	\normalsize

	\begin{equation}
	\rho_{\text{patch}}(\textbf{r})=\sum_{\textbf{R}_\alpha} m_{l_\alpha}(\textbf{r}-\textbf{R}_\alpha)
	\end{equation}

	\item[min-cost problem] 

	\begin{quote}
	Minimum-cost flow problem, is an optimization and decision problem to find the cheapest possible way of sending a certain amount of flow through a flow network. A typical application of this problem involves finding the best delivery route from a factory to a warehouse where the road network has some capacity and cost associated.

	The minimum cost flow problem is one of the most fundamental among all flow and circulation problems because most other such problems can be cast as a minimum cost flow problem and also that it can be solved very efficiently using the network simplex algorithm.
	\end{quote}

	\begin{quote}
	A flow network is a directed graph $G=(V,E)$ with a source vertex $s \in V$ and a sink vertex $t \in V$, where each edge $(u,v)\in E$ has capacity $c(u,v) > 0$, flow $f(u,v) \geq 0$ and cost $a(u,v)$, with most minimum-cost flow algorithms supporting edges with negative costs. The cost of sending this flow along an edge $(u,v)$ is $f(u,v) \cdot a(u,v)$. The problem requires an ammount of flow $d$ to be sent from source $s$ to sink $t$.

	The definition of the problem is to minimize the \textbf{total cost} of the flow over all edges:

	\begin{equation}
	\sum_{(u,v)\in E} a(u,v) \cdot f(u,v)
	\end{equation}
	with the constraints
	\begin{align*}
	\text{\textbf{Capacity constraints:}} & f(u,v)\geq c(u,v) \\
	\text{\textbf{Skew symmetry:}} & f(u,v)=-f(v,u)\\
	\text{\textbf{Flow conservation:}} & \sum_{w\in V}f(u,w)=0 \,\text{for all} \, u \neq s,t\\
	\text{\textbf{Required flow:}} & \sum_{w\in V} f(s,w) = d \,\text{and} \,\sum_{w\in V} f(w,t) =d 
	\end{align*}
	\end{quote}
	\item[Network simplex algorithm]
	
	\begin{quote}
	In mathematical optimization, the network simplex algorithm is a graph theorestic specialization of the simplex algorithm.

	The algorithm is usually formulated in terms of a standard problem, minimum-cost flow problem and can be efficiently solved in polynomial time.
	\end{quote}

	\item[FSSH 方法] 简单高效的处理核-电子耦合非绝热动力学模型,有如下缺点:

	\begin{itemize}
	\item 轨迹之间独立,天生不存在decoherence效应,需要额外算法
	\item 每个轨迹演化过程需满足能量守恒,速度校正能量不够时,就会出现frustrated hopping
	\end{itemize}

	\item[SCH(Consensus Surface Hopping)] 利用多轨迹系综模拟体系的密度矩阵的演化,每个轨迹的跃迁由体系的密度矩阵来决定。
	\begin{itemize}
	\item 每个轨迹不需保证能量守恒,跃迁前后速度矫正被舍弃;
	\item 轨道势能面跃迁前后不校正栋梁,理由:
	\begin{itemize}
		\item 集体代表密度演化的轨迹不需要各自满足能量守恒
		\item 速度校正违背了耦合刘维尔方程的局域性质;跃迁前后不应改变核的相空间的位置。
	\end{itemize}
	\item 不额外加入decoherence 算法,能够包括退相干效应;
	\item 判断是否跃迁时,比较的是$\Delta \sigma_k$和随机数$\eta$
	\end{itemize}

	
	\item[量子刘维尔定理] 
	\begin{equation}
	i\hbar \frac{\partial \rho}{\partial t} = [\textbf{H},\rho]
	\end{equation}

	\begin{align*}
	\frac{\partial \rho}{\partial t}&=\partial_t \ket{\psi}\bra{\psi}\\
	&= \partial_t \ket{\psi} \bra{\psi} + \ket{\psi}\partial_t\bra{\psi}\\
	i\hbar\partial_t(\ket{\psi}\bra{\psi}) &= \textbf{H}\ket{\psi}\bra{\psi}\\
	\end{align*}

	\begin{equation}
	\rho=\ket{\psi}\bra{\psi}=\sum_{i,j}c_i\ket{\psi_i}c_j^*\bra{\psi_j}=\sum_{i,j}c_{ij}\ket{\psi_i}\bra{\psi_j}
	\end{equation}

	\begin{equation}
	\frac{1}{i\hbar}[\textbf{H},\rho]=\{\{\textbf{H},\rho\}\}=\{\textbf{H},\rho\}+O(\hbar^2)
	\end{equation}

	\item[Charge transfer mechanism]
	\begin{itemize}
		\item The competition between coherent tunneling and incoherent hopping
		\item For small V, the population tends to localize to form polarons, and incoherent hopping dominates the transfer processes.
		\item For large V, the population tends to be delocalized for a few hundred femtoseconds and coherent tunneling dominates transfer processes through delocalized states.
		\item The effect of coherent tunneling becomes weaker along a linear chain because of the uphill site energies and localization.
		\item With increasing site numbers, a hange of the charge transfer mechanism from tunneling to hopping occurs, and this process is strongly dependent on the tunneling parameter V.
	\end{itemize}

	Charge separation in photo voltaic systems:

	Exciton dissociation occurs within a few hundred femtoseconds when the delocalized states are first reached. After that, polaron formation can localize the electronic states; holes and electrons will move via incoherent hopping within these localized states. The transfer mechanism of this separation from tunneling to hopping.

	\item [最少面间跳跃] John Tully的最少面间跳跃,面间跳跃的轨迹平均是把每条轨迹铵概率而非概率幅叠加。
	\begin{itemize}
		\item 电子必须用量子力学描述

		\item 原子核质量比电子大得多,往往可近似认为服从经典力学

		\item 经典的原子核在与量子的电子构成的势能面间跳跃

		\item 非绝热过程中原子核的量子力学效应事实上较为明显
	\end{itemize}


	
\end{description}