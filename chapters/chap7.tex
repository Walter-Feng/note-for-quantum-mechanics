\chapter{Gaussian Integrals}

All the knowledge here can be found in the article \emph{S. Obara and A. Saika, Efficient recursive computation of molecular integrals over Cartesian Gaussian functions, Journal of Chemical Physics 84(7),3963-3974 (1986)}. This chapter is a digest from \emph{Density Fitting in Explicitly Correlated Electronic Structure Theory} by \emph{Andrew James May}.

In the whole chapter the state is described as a gaussian function, and it is marked like this:
\begin{equation} 
	|a ) = g(\boldsymbol{r}, \alpha, \boldsymbol{a} ,\boldsymbol{A} ) = (x-A_x)^{a_x} (y-A_y)^{a_y} (z-A_z)^{a_z} \exp \left\{ - \alpha | \boldsymbol{r} - \boldsymbol{A} |^2  \right\}. 
\end{equation}  
But here a simple $| a \rangle $ is denoted. the  $\boldsymbol{a}$ represents the angular momentum exponents,
\[
	\boldsymbol{a} = (a_x,a_y,a_z)
.\] 

\begin{theorem}
	The product of two Gaussian functions,
	\[
		g_1 g_2 = g(\boldsymbol{r},\alpha, \boldsymbol{a},\boldsymbol{A}) g(\boldsymbol{r},\beta, \boldsymbol{b}, \boldsymbol{B})
	.\] 
	can be represented exactly as a small expansion in single Gaussian functions.
\end{theorem}
The simplest case of two Gaussian functions with zero angular momentum is expressed as 
	\[
		g(\boldsymbol{r},\alpha,\boldsymbol{0},\boldsymbol{A}) g(\boldsymbol{r},\beta, \boldsymbol{0}, \boldsymbol{B} ) = \exp\left\{ -\alpha | \boldsymbol{r}-\boldsymbol{A} | ^2 \right\} \exp \left\{ - \beta |\boldsymbol{r}- \boldsymbol{B}| ^2 \right\} 
	.\] 
	which can be algebraically rearranged to give
	\begin{align*}
		g(\boldsymbol{r},\alpha,\boldsymbol{0}, \boldsymbol{A}) g(\boldsymbol{r},\beta, \boldsymbol{0}, \boldsymbol{B}) &= \exp \left\{ - \xi |\bar{AB} |^2\right\}  \exp \left\{ - \zeta |\boldsymbol{r}-\boldsymbol{P}| ^2 \right\} \\
																&= exp \left\{ - \xi |\bar{AB}|^2 \right\} g(\boldsymbol{r},\zeta, \boldsymbol{0}, \boldsymbol{P})
	\end{align*}
	where
	\[
		\zeta = \alpha+ \beta, \xi = \frac{\alpha \beta}{\zeta}, \boldsymbol{P} = \frac{\alpha \boldsymbol{A} + \beta \boldsymbol{B}}{\zeta}, 
	.\] 
For a more general case,
\begin{align*}
	g_1 g_2 & = \exp \left( -\xi |\bar{AB} |^2  \right) \exp \left( -\zeta |\boldsymbol{r} - \boldsymbol{P} | ^2 \right) \times \\
		& \prod_{k=x,y,z}\left[ \sum_{i=0}^{a_k + b_k} (x-P_x)^i f_i(a_k, b_k, \boldsymbol{PA}_k,\boldsymbol{PB}_k)\right] 
\end{align*}

The first obvious step when attempting to combine the Gaussian functions is to substitute the exponential term
\[
	g_1 g_2 = \exp \left\{ -\xi |\bar{AB}| ^2  \right\} \exp \left\{ -\zeta |\boldsymbol{r}-\boldsymbol{P}| ^2 \right\} \times A(a_i,b_i)
.\] 
where $A(a_i,b_i)$ represents the angular part,
 \[
	 A(a_i, b_i) = \prod_{k=x,y,z} (k-A_k)^{a_k + b_k} 
.\] 
Terms $k-A_k)^{a_k}$ must be expressed using a binomial expansion
 \[
	 (x-A_x)^{a_x} = [ (x-P_x) + \boldsymbol{PA}_x]^{a_x} \sum_{i=0}^{a_x} 
	 \begin{pmatrix}
	 	a_x \\
		i
	 \end{pmatrix}
	 (x-P_x)^i \boldsymbol{PA}_x ^{a_x-i}
.\] 
and this allows everything to be shifted to the new center $\boldsymbol{P}$. Products of such terms of the same Cartesian coordinate can be thus be represented as follows
\[
\begin{aligned}
	(x-A_x)^{a_x} (x-B_x)^{b_x} &= \sum_{i=0}^{a_x}
	\begin{pmatrix}
		a_x \\
		i
	\end{pmatrix}
	(x-P_x)^i \boldsymbol{PA} ^{a_x - i}  \times \sum
	\begin{pmatrix}
		b_x \\
		j
	\end{pmatrix}
	(x-P_x)^j \boldsymbol{PB}_x ^{b_x-j} \\
				    &= \sum_0^{a_x + b_x} (x-P_x)^k f_k(a_x,b_x,\boldsymbol{PA}_x, \boldsymbol{PB}_x)
\end{aligned}
.\] 
where
\[
	f_k(a_x,b_x,\boldsymbol{PA}_x,\boldsymbol{PB}_x) = \sum_{i=0}^k 
	\begin{pmatrix}
		a_x \\
		i
	\end{pmatrix}
	\begin{pmatrix}
		b_x \\
		k-i
	\end{pmatrix}
	\boldsymbol{PA}_x^{a_x - i} \boldsymbol{PB}_x^{b_x - k + i}
.\] 


When applied to integrals the GPT can be used to reduce a four-index integral to a sum of three-index integrals
\[
	\langle ab | r_{12}^{-1} | cd \rangle = T_p^{ab} \langle p | r_{12}^{-1} | cd \rangle 
.\] 
where $p$ follows the Einstein summation convention, and
 \[
	 T_p^{ab} = \prod_{k=x,y,z} f_{p_k} (a_k,b_k,\boldsymbol{PA}_k, \boldsymbol{PA}_k) \exp \left\{ - \xi |\bar{AB}|^2 \right\} 
.\] 
and GPT being used once again,
\[
	\langle ab | r_{12}^{-1} | cd \rangle = T_p^{ab} t_q^{cd} \langle p | r_{12}^{-1} | q \rangle  
.\] 


Integrals of the form
\[
	\int ^\infty _{-\infty} dx (x-P_x)^i \exp \left\{ - \zeta (x-P_x)^2 \right\}  \, \mathrm{d}x = \int ^\infty_{-\infty} du u^i \exp \left\{ - \zeta u^2 \right\} \, \mathrm{d} u 
.\] 
and 
\[
	\int ^\infty_{-\infty}  u^i \exp \left\{ -\zeta u^2 \right\} \, \mathrm{d}u = \frac{(i-1)!!}{2\zeta^{\frac{i}{2}} }\left( \frac{\pi}{\zeta} \right) ^{\frac{1}{2}} 
.\] 
if $i$ is even and zero otherwise. 

Now all \hat{t} is needed is to evaluate
\[
\frac{1}{2\alpha} \frac{\partial }{\partial A_i}  \langle \boldsymbol{a} | \boldsymbol{b} \rangle 
.\] 
then
\[
	\frac{\partial }{\partial A_i} f_k (a_i,b_i,\boldsymbol{PA}_i, \boldsymbol{PB}_i) = - \frac{\beta}{\alpha+\beta} a_i f_k (a_i-1,b_i,\boldsymbol{PA}_i,\boldsymbol{PB}_i) - \frac{\alpha}{\alpha + \beta} b_i f_k (a_i, b_i -1 , \boldsymbol{PA}_i,\boldsymbol{PB}_i)
.\] 
The recurrence relation for increasing angular momentum for $| a \rangle $ is as
\[
	\langle a+1_i | b \rangle = (P_i - A_i) \langle a | b \rangle + \frac{a_i}{2 \zeta} \langle a- 1_i | b \rangle + \frac{b_i}{2\zeta} \langle a | b-1_i \rangle 
.\] 
where $1_i$ represents  $\delta_{ik}$, with $k$ being the flexible index.

The starting case for the recurrence relation is given by
\[
	\langle \boldsymbol{0}_A | \boldsymbol{0}_B \rangle = \left( \frac{\pi}{\zeta} \right) ^{\frac{3}{2}} \exp \left\{ - \xi | \bar{AB} | ^2 \right\} 
.\] 

The Coulomb integrals are defined as
\[
	\langle a | \frac{1}{r_{12}} | b \rangle = \int \int g(\boldsymbol{r}_1,\alpha,\boldsymbol{a},\boldsymbol{A}) \frac{1}{r_{12}} g(\boldsymbol{r}_2,\beta, \boldsymbol{b},\boldsymbol{B}) \, \mathrm{d}\boldsymbol{r}_2  \, \mathrm{d} \boldsymbol{r}_1 
.\] 
The recursion relation is given by
\[
	\langle a + 1_i | \frac{1}{r_{12}} | b \rangle = \boldsymbol{PA}_i \langle a | \frac{1}{r_{12}} | b \rangle ^{(m+1)} + \frac{a_i}{2\alpha} \left\{ \langle a-1_i | \frac{1}{r_{12}} | b \rangle ^{(m)} - \frac{\beta}{\zeta}\langle a - 1_i | \frac{1}{r_{12}} | b \rangle ^{(m+1)}\right\} + \frac{b_i}{2\xi} \langle a | \frac{1}{r_{12}} | b - 1_i \rangle ^{(m+1)} 
.\] 
where
\[
\xi = \frac{\alpha \beta}{\zeta}
.\] 
The starting case for the recurrence relation is given by
\[
	\langle s | \frac{1}{r_{12}} | s \rangle ^{(m)} = \frac{2\pi ^{\frac{5}{2}}}{\alpha \beta \frac{\zeta^1}{2}} F_m(T)
.\] 
where
\[
	T = \xi |\bar{AB} | ^2
.\] 
and 
\[
	F_m (T) = \int ^1_0 t^{2m} \exp \left\{ - T t^2 \right\}  \, \mathrm{d}t 
.\] 
The integral denoted by the function $F_m(T)$ is known as the Boys function and is evaluated through standard methods.



