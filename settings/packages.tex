\usepackage{graphicx}
\usepackage{subfigure}
\usepackage{multirow}
\usepackage{array}
\usepackage{booktabs} % Top and bottom rules for table不【
\usepackage[font=small,labelfont=bf]{caption} % Required for specifying captions to tables and figures
\usepackage{amsfonts, amsmath, amssymb} % For math fonts, symbols and environments
\usepackage{wrapfig} % Allows wrapping text around tables and figures
\usepackage[centering]{geometry}
\geometry{top=25.4mm, bottom=25.4mm, left=25.4mm, right=25.4mm}
\usepackage[normalem]{ulem}
\usepackage{titlesec}
\usepackage{xeCJK}
\setCJKmainfont[ItalicFont={KaiTi}, BoldFont={SimHei}]{KaiTi}
\usepackage{textcomp}
\usepackage[final]{pdfpages}
\usepackage{chemfig}
\usepackage{fontspec}
\usepackage{xunicode,xltxtra}
\usepackage{setspace}
\usepackage{listings}
\usepackage{xcolor} % 定制颜色
\definecolor{mygreen}{rgb}{0,0.6,0}
\definecolor{mygray}{rgb}{0.5,0.5,0.5}
\definecolor{mymauve}{rgb}{0.58,0,0.82}
\lstset{
backgroundcolor=\color{white},      % choose the background color
basicstyle=\footnotesize\ttfamily,  % size of fonts used for the code
columns=fullflexible,
tabsize=4,
breaklines=true,               % automatic line breaking only at whitespace
captionpos=b,                  % sets the caption-position to bottom
commentstyle=\color{mygreen},  % comment style
escapeinside={\%*}{*)},        % if you want to add LaTeX within your code
keywordstyle=\color{blue},     % keyword style
stringstyle=\color{mymauve}\ttfamily,  % string literal style
frame=single,
rulesepcolor=\color{red!20!green!20!blue!20},
% identifierstyle=\color{red},
language=Mathematica,
}


\newtheorem{theorem}{Theorem}[section]
\newtheorem{lemma}{Lemma}  
\newtheorem{definition}{Definition}[section]
\newtheorem{proof}{Proof}[section]  
\numberwithin{equation}{section}

\newcommand{\bra}[1]{\langle #1 |}
\newcommand{\ket}[1]{| #1 \rangle}
\newcommand{\bracket}[2]{\langle #1 | #2 \rangle}
\newcommand{\bracketl}[3]{\langle #1 | #2 | #3 \rangle}
\newcommand{\func}{\mathrm \,}
\newcommand{\define}[2]{
	\begin{definition}
	\begin{description}
	\item[#1]
	#2
	\end{description}
	\end{definition}
}
\newcommand{\mean}[1]{\langle #1 \rangle}

\newcommand{\sch}{Schr\"odinger}
\newcommand{\grad}{\nabla}
\newcommand{\ueq}{\neq}

\newcommand{\p}{\textbf{p}}
\newcommand{\x}{\textbf{x}}

\newcommand{\y}{\textbf{y}}
\newcommand{\z}{\textbf{z}}